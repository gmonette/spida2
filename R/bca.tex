% Options for packages loaded elsewhere
\PassOptionsToPackage{unicode}{hyperref}
\PassOptionsToPackage{hyphens}{url}
%
\documentclass[
]{article}
\usepackage{amsmath,amssymb}
\usepackage{iftex}
\ifPDFTeX
  \usepackage[T1]{fontenc}
  \usepackage[utf8]{inputenc}
  \usepackage{textcomp} % provide euro and other symbols
\else % if luatex or xetex
  \usepackage{unicode-math} % this also loads fontspec
  \defaultfontfeatures{Scale=MatchLowercase}
  \defaultfontfeatures[\rmfamily]{Ligatures=TeX,Scale=1}
\fi
\usepackage{lmodern}
\ifPDFTeX\else
  % xetex/luatex font selection
\fi
% Use upquote if available, for straight quotes in verbatim environments
\IfFileExists{upquote.sty}{\usepackage{upquote}}{}
\IfFileExists{microtype.sty}{% use microtype if available
  \usepackage[]{microtype}
  \UseMicrotypeSet[protrusion]{basicmath} % disable protrusion for tt fonts
}{}
\makeatletter
\@ifundefined{KOMAClassName}{% if non-KOMA class
  \IfFileExists{parskip.sty}{%
    \usepackage{parskip}
  }{% else
    \setlength{\parindent}{0pt}
    \setlength{\parskip}{6pt plus 2pt minus 1pt}}
}{% if KOMA class
  \KOMAoptions{parskip=half}}
\makeatother
\usepackage{xcolor}
\usepackage[margin=1in]{geometry}
\usepackage{color}
\usepackage{fancyvrb}
\newcommand{\VerbBar}{|}
\newcommand{\VERB}{\Verb[commandchars=\\\{\}]}
\DefineVerbatimEnvironment{Highlighting}{Verbatim}{commandchars=\\\{\}}
% Add ',fontsize=\small' for more characters per line
\usepackage{framed}
\definecolor{shadecolor}{RGB}{248,248,248}
\newenvironment{Shaded}{\begin{snugshade}}{\end{snugshade}}
\newcommand{\AlertTok}[1]{\textcolor[rgb]{0.94,0.16,0.16}{#1}}
\newcommand{\AnnotationTok}[1]{\textcolor[rgb]{0.56,0.35,0.01}{\textbf{\textit{#1}}}}
\newcommand{\AttributeTok}[1]{\textcolor[rgb]{0.13,0.29,0.53}{#1}}
\newcommand{\BaseNTok}[1]{\textcolor[rgb]{0.00,0.00,0.81}{#1}}
\newcommand{\BuiltInTok}[1]{#1}
\newcommand{\CharTok}[1]{\textcolor[rgb]{0.31,0.60,0.02}{#1}}
\newcommand{\CommentTok}[1]{\textcolor[rgb]{0.56,0.35,0.01}{\textit{#1}}}
\newcommand{\CommentVarTok}[1]{\textcolor[rgb]{0.56,0.35,0.01}{\textbf{\textit{#1}}}}
\newcommand{\ConstantTok}[1]{\textcolor[rgb]{0.56,0.35,0.01}{#1}}
\newcommand{\ControlFlowTok}[1]{\textcolor[rgb]{0.13,0.29,0.53}{\textbf{#1}}}
\newcommand{\DataTypeTok}[1]{\textcolor[rgb]{0.13,0.29,0.53}{#1}}
\newcommand{\DecValTok}[1]{\textcolor[rgb]{0.00,0.00,0.81}{#1}}
\newcommand{\DocumentationTok}[1]{\textcolor[rgb]{0.56,0.35,0.01}{\textbf{\textit{#1}}}}
\newcommand{\ErrorTok}[1]{\textcolor[rgb]{0.64,0.00,0.00}{\textbf{#1}}}
\newcommand{\ExtensionTok}[1]{#1}
\newcommand{\FloatTok}[1]{\textcolor[rgb]{0.00,0.00,0.81}{#1}}
\newcommand{\FunctionTok}[1]{\textcolor[rgb]{0.13,0.29,0.53}{\textbf{#1}}}
\newcommand{\ImportTok}[1]{#1}
\newcommand{\InformationTok}[1]{\textcolor[rgb]{0.56,0.35,0.01}{\textbf{\textit{#1}}}}
\newcommand{\KeywordTok}[1]{\textcolor[rgb]{0.13,0.29,0.53}{\textbf{#1}}}
\newcommand{\NormalTok}[1]{#1}
\newcommand{\OperatorTok}[1]{\textcolor[rgb]{0.81,0.36,0.00}{\textbf{#1}}}
\newcommand{\OtherTok}[1]{\textcolor[rgb]{0.56,0.35,0.01}{#1}}
\newcommand{\PreprocessorTok}[1]{\textcolor[rgb]{0.56,0.35,0.01}{\textit{#1}}}
\newcommand{\RegionMarkerTok}[1]{#1}
\newcommand{\SpecialCharTok}[1]{\textcolor[rgb]{0.81,0.36,0.00}{\textbf{#1}}}
\newcommand{\SpecialStringTok}[1]{\textcolor[rgb]{0.31,0.60,0.02}{#1}}
\newcommand{\StringTok}[1]{\textcolor[rgb]{0.31,0.60,0.02}{#1}}
\newcommand{\VariableTok}[1]{\textcolor[rgb]{0.00,0.00,0.00}{#1}}
\newcommand{\VerbatimStringTok}[1]{\textcolor[rgb]{0.31,0.60,0.02}{#1}}
\newcommand{\WarningTok}[1]{\textcolor[rgb]{0.56,0.35,0.01}{\textbf{\textit{#1}}}}
\usepackage{graphicx}
\makeatletter
\newsavebox\pandoc@box
\newcommand*\pandocbounded[1]{% scales image to fit in text height/width
  \sbox\pandoc@box{#1}%
  \Gscale@div\@tempa{\textheight}{\dimexpr\ht\pandoc@box+\dp\pandoc@box\relax}%
  \Gscale@div\@tempb{\linewidth}{\wd\pandoc@box}%
  \ifdim\@tempb\p@<\@tempa\p@\let\@tempa\@tempb\fi% select the smaller of both
  \ifdim\@tempa\p@<\p@\scalebox{\@tempa}{\usebox\pandoc@box}%
  \else\usebox{\pandoc@box}%
  \fi%
}
% Set default figure placement to htbp
\def\fps@figure{htbp}
\makeatother
\setlength{\emergencystretch}{3em} % prevent overfull lines
\providecommand{\tightlist}{%
  \setlength{\itemsep}{0pt}\setlength{\parskip}{0pt}}
\setcounter{secnumdepth}{-\maxdimen} % remove section numbering
\usepackage{bookmark}
\IfFileExists{xurl.sty}{\usepackage{xurl}}{} % add URL line breaks if available
\urlstyle{same}
\hypersetup{
  pdftitle={bca.R},
  pdfauthor={georges},
  hidelinks,
  pdfcreator={LaTeX via pandoc}}

\title{bca.R}
\author{georges}
\date{2025-09-15}

\begin{document}
\maketitle

Bias-corrected and accelerated confidence intervals

This function uses the method proposed by DiCiccio and Efron (1996) to
generate confidence intervals that produce more accurate coverage rates
when the distribution of bootstrap draws is non-normal. This code is
adapted from the \code{BC.CI()} function within the
\code{\link[mediation]{mediate}} function in the \code{mediation}
package.

@param theta a vector that contains draws of a quantity of interest
using bootstrap samples. The length of \code{theta} is equal to the
number of iterations in the previously-run bootstrap simulation. @param
conf.level the level of the desired confidence interval, as a
proportion. Defaults to .95 which returns the 95 percent confidence
interval.

@details \eqn{BC_a} confidence intervals are typically calculated using
influence statistics from jackknife simulations. For our purposes,
however, running jackknife simulation in addition to ordinary
bootstrapping is too computationally expensive. This function follows
the procedure outlined by DiCiccio and Efron (1996, p.~201) to calculate
the bias-correction and acceleration parameters using only the draws
from ordinary bootstrapping. @return returns a vector of length 2 in
which the first element is the lower bound and the second element is the
upper bound

@author Jonathan Kropko \textless jkropko@@virginia.edu\textgreater{}
and Jeffrey J. Harden \textless jharden@@nd.edu\textgreater, based on
the code for the \code{\link[mediation]{mediate}} function in the
\code{mediation} package by Dustin Tingley, Teppei Yamamoto, Kentaro
Hirose, Luke Keele, and Kosuke Imai. @references DiCiccio, T. J. and B.
Efron. (1996). Bootstrap Confidence Intervals.
\emph{Statistical Science}. 11(3): 189--212.
\url{https://doi.org/10.1214/ss/1032280214} @seealso
\code{\link[coxed]{coxed}}, \code{\link[rms]{bootcov}},
\code{\link[mediation]{mediate}} @export @examples theta \textless-
rnorm(1000, mean=3, sd=4) bca(theta, conf.level = .95) \# \# Why
confidence intervals types matter: \# bca(rnorm(1000))
bca(exp(rnorm(1000)))

\begin{Shaded}
\begin{Highlighting}[]
\NormalTok{bca }\OtherTok{\textless{}{-}} \ControlFlowTok{function}\NormalTok{(theta, }\AttributeTok{conf.level =}\NormalTok{ .}\DecValTok{95}\NormalTok{)\{}

  \ControlFlowTok{if}\NormalTok{(}\FunctionTok{var}\NormalTok{(theta)}\SpecialCharTok{==}\DecValTok{0}\NormalTok{)\{}
\NormalTok{    lower }\OtherTok{\textless{}{-}} \FunctionTok{mean}\NormalTok{(theta)}
\NormalTok{    upper }\OtherTok{\textless{}{-}} \FunctionTok{mean}\NormalTok{(theta)}
    \FunctionTok{return}\NormalTok{(}\FunctionTok{c}\NormalTok{(lower, upper))}
\NormalTok{  \}}

  \ControlFlowTok{if}\NormalTok{(}\FunctionTok{max}\NormalTok{(theta)}\SpecialCharTok{==}\ConstantTok{Inf} \SpecialCharTok{|} \FunctionTok{min}\NormalTok{(theta)}\SpecialCharTok{=={-}}\ConstantTok{Inf}\NormalTok{)\{}
    \FunctionTok{stop}\NormalTok{(}\StringTok{"bca() function does not work when some values are infinite"}\NormalTok{)}
\NormalTok{  \}}

\NormalTok{  low }\OtherTok{\textless{}{-}}\NormalTok{ (}\DecValTok{1} \SpecialCharTok{{-}}\NormalTok{ conf.level)}\SpecialCharTok{/}\DecValTok{2}
\NormalTok{  high }\OtherTok{\textless{}{-}} \DecValTok{1} \SpecialCharTok{{-}}\NormalTok{ low}
\NormalTok{  sims }\OtherTok{\textless{}{-}} \FunctionTok{length}\NormalTok{(theta)}
\NormalTok{  z.inv }\OtherTok{\textless{}{-}} \FunctionTok{length}\NormalTok{(theta[theta }\SpecialCharTok{\textless{}} \FunctionTok{mean}\NormalTok{(theta)])}\SpecialCharTok{/}\NormalTok{sims}
\NormalTok{  z }\OtherTok{\textless{}{-}} \FunctionTok{qnorm}\NormalTok{(z.inv)}
\NormalTok{  U }\OtherTok{\textless{}{-}}\NormalTok{ (sims }\SpecialCharTok{{-}} \DecValTok{1}\NormalTok{) }\SpecialCharTok{*}\NormalTok{ (}\FunctionTok{mean}\NormalTok{(theta, }\AttributeTok{na.rm=}\ConstantTok{TRUE}\NormalTok{) }\SpecialCharTok{{-}}\NormalTok{ theta)}
\NormalTok{  top }\OtherTok{\textless{}{-}} \FunctionTok{sum}\NormalTok{(U}\SpecialCharTok{\^{}}\DecValTok{3}\NormalTok{)}
\NormalTok{  under }\OtherTok{\textless{}{-}} \DecValTok{6} \SpecialCharTok{*}\NormalTok{ (}\FunctionTok{sum}\NormalTok{(U}\SpecialCharTok{\^{}}\DecValTok{2}\NormalTok{))}\SpecialCharTok{\^{}}\NormalTok{\{}\DecValTok{3}\SpecialCharTok{/}\DecValTok{2}\NormalTok{\}}
\NormalTok{  a }\OtherTok{\textless{}{-}}\NormalTok{ top }\SpecialCharTok{/}\NormalTok{ under}
\NormalTok{  lower.inv }\OtherTok{\textless{}{-}}  \FunctionTok{pnorm}\NormalTok{(z }\SpecialCharTok{+}\NormalTok{ (z }\SpecialCharTok{+} \FunctionTok{qnorm}\NormalTok{(low))}\SpecialCharTok{/}\NormalTok{(}\DecValTok{1} \SpecialCharTok{{-}}\NormalTok{ a }\SpecialCharTok{*}\NormalTok{ (z }\SpecialCharTok{+} \FunctionTok{qnorm}\NormalTok{(low))))}
\NormalTok{  lower }\OtherTok{\textless{}{-}} \FunctionTok{quantile}\NormalTok{(theta, lower.inv, }\AttributeTok{names=}\ConstantTok{FALSE}\NormalTok{)}
\NormalTok{  upper.inv }\OtherTok{\textless{}{-}}  \FunctionTok{pnorm}\NormalTok{(z }\SpecialCharTok{+}\NormalTok{ (z }\SpecialCharTok{+} \FunctionTok{qnorm}\NormalTok{(high))}\SpecialCharTok{/}\NormalTok{(}\DecValTok{1} \SpecialCharTok{{-}}\NormalTok{ a }\SpecialCharTok{*}\NormalTok{ (z }\SpecialCharTok{+} \FunctionTok{qnorm}\NormalTok{(high))))}
\NormalTok{  upper }\OtherTok{\textless{}{-}} \FunctionTok{quantile}\NormalTok{(theta, upper.inv, }\AttributeTok{names=}\ConstantTok{FALSE}\NormalTok{)}
  
\NormalTok{  ret }\OtherTok{\textless{}{-}} \FunctionTok{data.frame}\NormalTok{(}
    \AttributeTok{lower =} \FunctionTok{c}\NormalTok{(lower, }\FunctionTok{mean}\NormalTok{(theta) }\SpecialCharTok{{-}} \FunctionTok{qnorm}\NormalTok{(high)}\SpecialCharTok{*}\FunctionTok{sd}\NormalTok{(theta), }\FunctionTok{quantile}\NormalTok{(theta,low)),}
    \AttributeTok{upper =} \FunctionTok{c}\NormalTok{(upper, }\FunctionTok{mean}\NormalTok{(theta) }\SpecialCharTok{+} \FunctionTok{qnorm}\NormalTok{(high)}\SpecialCharTok{*}\FunctionTok{sd}\NormalTok{(theta), }\FunctionTok{quantile}\NormalTok{(theta,high))}
\NormalTok{  )}
  \FunctionTok{rownames}\NormalTok{(ret) }\OtherTok{\textless{}{-}} \FunctionTok{c}\NormalTok{(}\StringTok{\textquotesingle{}bca(mod)\textquotesingle{}}\NormalTok{,}\StringTok{\textquotesingle{}normal\textquotesingle{}}\NormalTok{,}\StringTok{\textquotesingle{}percentile\textquotesingle{}}\NormalTok{)}
  \FunctionTok{attr}\NormalTok{(ret,}\StringTok{\textquotesingle{}conf.level\textquotesingle{}}\NormalTok{) }\OtherTok{\textless{}{-}} \FunctionTok{c}\NormalTok{(}\AttributeTok{conf.level=}\NormalTok{conf.level) }
  \FunctionTok{class}\NormalTok{(ret) }\OtherTok{\textless{}{-}} \FunctionTok{c}\NormalTok{(}\StringTok{\textquotesingle{}conf\_interval\textquotesingle{}}\NormalTok{, }\FunctionTok{class}\NormalTok{(ret))}
\NormalTok{  ret}
\NormalTok{\}}
\end{Highlighting}
\end{Shaded}

@export

\begin{Shaded}
\begin{Highlighting}[]
\NormalTok{print.conf\_interval }\OtherTok{\textless{}{-}} \ControlFlowTok{function}\NormalTok{(x,...) \{}
  \FunctionTok{NextMethod}\NormalTok{()}
  \FunctionTok{print}\NormalTok{(}\FunctionTok{attr}\NormalTok{(x,}\StringTok{\textquotesingle{}conf.level\textquotesingle{}}\NormalTok{))}
\NormalTok{\}}
\end{Highlighting}
\end{Shaded}


\end{document}
